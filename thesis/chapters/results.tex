\section{Results}

We evaluated the performance of small language models across three different prompting configurations on a dataset of medical discharge reports. The goal was to identify whether specific sections (e.g., diagnoses) were complete, missing, or improperly inferred. Each setup was tested on a standardized set of reports, and outputs were manually annotated against reference ground truth to classify model responses into true positives (TP), false negatives (FN), and irrelevant completions (IR).

\subsection{Evaluation Setups}

\textbf{Setup A}: Baseline prompt applied to DOCX reports converted to plain text using \texttt{python-docx}.  
\textbf{Setup B}: Baseline prompt applied to manually formatted markdown versions of the same reports.  
\textbf{Setup C}: Agentic prompting framework that invokes tool-calling for diagnosis section analysis.

\subsection{Quantitative Metrics}

Figure~\ref{fig:barplot} shows the distribution of evaluation outcomes (TP, FN, IR) across the three setups. Setup C significantly reduces false negatives while maintaining a low rate of irrelevant completions, indicating improved alignment with the prompt under tool-assisted reasoning.

\begin{figure}[H]
  \centering
  \includegraphics[width=0.8\textwidth]{barplot_results.png}
  \caption{Evaluation outcome counts for each setup.}
  \label{fig:barplot}
\end{figure}

\subsection{Section-Level Comparison}

To evaluate performance on a per-section basis, we analyzed accuracy for the diagnosis section in detail, comparing the precision of model outputs in detecting missing content. Figure~\ref{fig:diagnosis_precision} illustrates per-model precision.

\begin{figure}[H]
  \centering
  \includegraphics[width=0.8\textwidth]{diagnosis_precision.png}
  \caption{Precision for detecting incomplete diagnosis sections.}
  \label{fig:diagnosis_precision}
\end{figure}

\subsection{Prompt Adherence and Hallucination Rate}

We also measured the tendency of each model to diverge from prompt instructions, such as adding non-existent information or hallucinating sections. Figure~\ref{fig:irrelevant_rate} visualizes the rate of irrelevant completions per setup.

\begin{figure}[H]
  \centering
  \includegraphics[width=0.8\textwidth]{irrelevant_completions.png}
  \caption{Percentage of completions classified as irrelevant.}
  \label{fig:irrelevant_rate}
\end{figure}

