\section{Conclustion}
This thesis investigated whether small, open-source language models can be optimized to evaluate the completeness of medical discharge reports in a manner comparable to proprietary systems such as ChatGPT. The results show that such models—particularly those in the 7–8B parameter range—can be effectively adapted to perform section-level completeness evaluation, especially when guided by carefully designed prompts and lightweight agentic workflows.

However, this performance was demonstrated under a simplified evaluation framework based on binary classification (complete vs. incomplete) for individual report sections. While the approach enables clear metric comparison and model alignment, it does not yet capture the full nuance of clinical document quality, such as partial completeness, semantic adequacy, or medical relevance.

Future work should extend this framework to more complex classification schemes and richer annotation layers, enabling models not only to detect missing content but also to assess clinical plausibility, redundancy, or contradiction. Nonetheless, the findings presented here highlight the potential of small, local LLMs in supporting structured clinical documentation analysis—particularly in privacy-sensitive or resource-constrained environments.

